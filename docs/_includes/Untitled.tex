\documentclass[11pt, a4paper]{article}
%\usepackage{geometry}
\usepackage[inner=1.5cm,outer=1.5cm,top=2.5cm,bottom=2.5cm]{geometry}
\pagestyle{empty}
\usepackage{graphicx}
\usepackage{fancyhdr, lastpage, bbding, pmboxdraw}
\usepackage[usenames,dvipsnames]{color}
\definecolor{darkblue}{rgb}{0,0,.6}
\definecolor{darkred}{rgb}{.7,0,0}
\definecolor{darkgreen}{rgb}{0,.6,0}
\definecolor{red}{rgb}{.98,0,0}
\usepackage[colorlinks,pagebackref,pdfusetitle,urlcolor=darkblue,citecolor=darkblue,linkcolor=darkred,bookmarksnumbered,plainpages=false]{hyperref}
\renewcommand{\thefootnote}{\fnsymbol{footnote}}

\pagestyle{fancyplain}
\fancyhf{}
\lhead{ \fancyplain{}{Course Name} }
%\chead{ \fancyplain{}{} }
\rhead{ \fancyplain{}{\today} }
%\rfoot{\fancyplain{}{page \thepage\ of \pageref{LastPage}}}
\fancyfoot[RO, LE] {page \thepage\ of \pageref{LastPage} }
\thispagestyle{plain}

%%%%%%%%%%%% LISTING %%%
\usepackage{listings}
\usepackage{caption}
\DeclareCaptionFont{white}{\color{white}}
\DeclareCaptionFormat{listing}{\colorbox{gray}{\parbox{\textwidth}{#1#2#3}}}
\captionsetup[lstlisting]{format=listing,labelfont=white,textfont=white}
\usepackage{verbatim} % used to display code
\usepackage{fancyvrb}
\usepackage{acronym}
\usepackage{amsthm}
\VerbatimFootnotes % Required, otherwise verbatim does not work in footnotes!



\definecolor{OliveGreen}{cmyk}{0.64,0,0.95,0.40}
\definecolor{CadetBlue}{cmyk}{0.62,0.57,0.23,0}
\definecolor{lightlightgray}{gray}{0.93}



\lstset{
%language=bash,                          % Code langugage
basicstyle=\ttfamily,                   % Code font, Examples: \footnotesize, \ttfamily
keywordstyle=\color{OliveGreen},        % Keywords font ('*' = uppercase)
commentstyle=\color{gray},              % Comments font
numbers=left,                           % Line nums position
numberstyle=\tiny,                      % Line-numbers fonts
stepnumber=1,                           % Step between two line-numbers
numbersep=5pt,                          % How far are line-numbers from code
backgroundcolor=\color{lightlightgray}, % Choose background color
frame=none,                             % A frame around the code
tabsize=2,                              % Default tab size
captionpos=t,                           % Caption-position = bottom
breaklines=true,                        % Automatic line breaking?
breakatwhitespace=false,                % Automatic breaks only at whitespace?
showspaces=false,                       % Dont make spaces visible
showtabs=false,                         % Dont make tabls visible
columns=flexible,                       % Column format
morekeywords={__global__, __device__},  % CUDA specific keywords
}

%%% Работа с русским языком
\usepackage{cmap}					% поиск в PDF
\usepackage{mathtext} 				% русские буквы в формулах
\usepackage[T2A]{fontenc}			% кодировка
\usepackage[utf8]{inputenc}			% кодировка исходного текста
\usepackage[english,russian]{babel}	% локализация и переносы

%%%%%%%%%%%%%%%%%%%%%%%%%%%%%%%%%%%%
\begin{document}

\begin{center}
{\Large \textsc{Макроэкономика --- 2}}
\end{center}
\begin{center}
Осень 2018
\end{center}
%\date{September 26, 2014}

\begin{center}
\rule{6in}{0.4pt}
\begin{minipage}[t]{.75\textwidth}
\begin{tabular}{llcccll}
\textbf{Лектор:} & Кирилл Сосунов & & &  & \textbf{Email:} & \href{mailto:XYZ@email.org}{XYZ@email.org} \\
\textbf{Семинарист:} &  Диана & & & & \textbf{Email:} & \href{mailto:XYZ@email.org}{XYZ@email.org} \\ 
\textbf{Семинарист:} &  Александр Тишин & & & & \textbf{Email:} & \href{mailto:XYZ@email.org}{atishin@nes.ru}
\end{tabular}
\end{minipage}
\rule{6in}{0.4pt}
\end{center}
\vspace{.5cm}
\setlength{\unitlength}{1in}
\renewcommand{\arraystretch}{2}

\noindent\textbf{Страницы курса:} \begin{enumerate}
\item \url{macro.cf}
\item \url{https://phenyard.github.io/macro201/}
\end{enumerate}

\vskip.15in
\noindent\textbf{Office Hours:} После занятия, либо по предварительной договоренности. 

\vskip.15in
\noindent\textbf{Литература:} %\footnotemark
This is a  restricted list of various interesting and useful books that will be touched during the course. You need to consult them occasionally.
\begin{itemize}
\item Christopher M. Bishop, {\textit{Pattern Recognition and Machine Learning}}, Springer, 2006.
\item Peter J. Carrington, John Scott, and Stanley Wasserman, {\textit{Models and Methods in Social Network Analysis}}, Cambridge University Press, 2005.
\item Richard O. Duda, Peter E. Hart, and David G. Stork, {\textit{Pattern Classification}}, Wiley, 2nd ed., 2000.
\item Peter Flach, {\textit{Machine Learning: The Art and Science of Algorithms that Make Sense of Data}}, Cambridge University Press, 2012.

\end{itemize} 

% \footnotetext{Downloadable ebook versions are available on AeLP.}

\vskip.15in
\noindent\textbf{Цели курса:}  This course is  primarily designed for graduate students ... 




\vspace*{.15in}

\noindent \textbf{Предварительный план курса:}
\begin{center} 
\begin{minipage}{5in}
\begin{flushleft}
%Chapter 1 \dotfill ~$\approx$ 3 days \\
{\color{darkgreen}{\Rectangle}} ~A little of probability theory and graph theory	
\end{flushleft}
\end{minipage}
\end{center}

\vspace*{.15in}
\noindent\textbf{Оценки:}  
\begin{itemize}
\item 10 \% Домашние задания (4 штуки, раз в 2-3 недели, дедлайн каждой домашней работы 1 неделя (до следующего семинара)
\item 10 \% Десятиминутные тесты в классе (4 штуки, даются без предупреждения)
\item  30 \% Контрольная работа 
\item 50 \% Финальный экзамен  
\end{itemize}
ИТОГОВАЯ? 

\vskip.15in
\noindent\textbf{Важные даты:}
\begin{center} \begin{minipage}{3.8in}
\begin{flushleft}
Midterm \#1      \dotfill ~\={A}b\={a}n 16, 1393  \\
Final Exam       \dotfill ~Dey 18, 1393  \\
\end{flushleft}
\end{minipage}
\end{center}




\vskip.15in
\noindent\textbf{Посещение:}  Посещение лекций и семинаров необходимо и ожидается от студентов. 

\vskip.15in
\noindent\textbf{Академическая этика:}  Обман, плагиат, списывание и любые другие нарушения академической этики недопустимы.


%%%%%% THE END 
\end{document} 
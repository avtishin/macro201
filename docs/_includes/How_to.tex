% Этот шаблон документа разработан в 2014 году
% Данилом Фёдоровых (danil@fedorovykh.ru) 
% для использования в курсе 
% <<Документы и презентации в \LaTeX>>, записанном НИУ ВШЭ
% для Coursera.org: http://coursera.org/course/latex .
% Текущая версия шаблона — октябрь 2015 г.
% Исходная версия шаблона --- 
% https://www.writelatex.com/coursera/latex/5.2.1

\documentclass[a4paper,12pt]{article}

%%% Работа с русским языком
\usepackage{cmap}					% поиск в PDF
\usepackage{mathtext} 				% русские буквы в формулах
\usepackage[T2A]{fontenc}			% кодировка
\usepackage[utf8]{inputenc}			% кодировка исходного текста
\usepackage[english,russian]{babel}	% локализация и переносы
\usepackage{indentfirst} %Абзац после заголовка
\frenchspacing %"Французские" пробелы


%%% Дополнительная работа с математикой
\usepackage{amsmath,amsfonts,amssymb,amsthm,mathtools} % AMS
\usepackage{icomma} % "Умная" запятая: $0,2$ --- число, $0, 2$ --- перечисление

%% Номера формул
%\mathtoolsset{showonlyrefs=true} % Показывать номера только у тех формул, на которые есть \eqref{} в тексте.
%\usepackage{leqno} % Нумерация формул слева

%% Свои команды
\DeclareMathOperator{\sgn}{\mathop{sgn}}

%% Перенос знаков в формулах (по Львовскому)
\newcommand*{\hm}[1]{#1\nobreak\discretionary{}
{\hbox{$\mathsurround=0pt #1$}}{}}

%%% Работа с картинками
\usepackage{graphicx}  % Для вставки рисунков
\graphicspath{{images/}{images2/}}  % папки с картинками
\setlength\fboxsep{3pt} % Отступ рамки \fbox{} от рисунка
\setlength\fboxrule{1pt} % Толщина линий рамки \fbox{}
\usepackage{wrapfig} % Обтекание рисунков текстом

%%% Работа с таблицами
\usepackage{array,tabularx,tabulary,booktabs} % Дополнительная работа с таблицами
\usepackage{longtable}  % Длинные таблицы
\usepackage{multirow} % Слияние строк в таблице

%%% Теоремы
\theoremstyle{plain} % Это стиль по умолчанию, его можно не переопределять.
\newtheorem{theorem}{Теорема}[section]
\newtheorem{proposition}[theorem]{Утверждение}
 
\theoremstyle{definition} % "Определение"
\newtheorem{corollary}{Следствие}[theorem]
\newtheorem{problem}{Задача}[section]
 
\theoremstyle{remark} % "Примечание"
\newtheorem*{nonum}{Решение}

%%% Программирование
\usepackage{etoolbox} % логические операторы

%%% Страница
\usepackage{extsizes} % Возможность сделать 14-й шрифт
\usepackage{geometry} % Простой способ задавать поля
	\geometry{top=25mm}
	\geometry{bottom=35mm}
	\geometry{left=35mm}
	\geometry{right=20mm}
 %
%\usepackage{fancyhdr} % Колонтитулы
% 	\pagestyle{fancy}
 	%\renewcommand{\headrulewidth}{0pt}  % Толщина линейки, отчеркивающей верхний колонтитул
% 	\lfoot{Нижний левый}
% 	\rfoot{Нижний правый}
% 	\rhead{Верхний правый}
% 	\chead{Верхний в центре}
% 	\lhead{Верхний левый}
%	\cfoot{Нижний в центре} % По умолчанию здесь номер страницы

\usepackage{setspace} % Интерлиньяж
%\onehalfspacing % Интерлиньяж 1.5
%\doublespacing % Интерлиньяж 2
%\singlespacing % Интерлиньяж 1

\usepackage{lastpage} % Узнать, сколько всего страниц в документе.

\usepackage{soul} % Модификаторы начертания

\usepackage{hyperref}
\usepackage[usenames,dvipsnames,svgnames,table,rgb]{xcolor}
\hypersetup{				% Гиперссылки
    unicode=true,           % русские буквы в раздела PDF
    pdftitle={Заголовок},   % Заголовок
    pdfauthor={Автор},      % Автор
    pdfsubject={Тема},      % Тема
    pdfcreator={Создатель}, % Создатель
    pdfproducer={Производитель}, % Производитель
    pdfkeywords={keyword1} {key2} {key3}, % Ключевые слова
    colorlinks=true,       	% false: ссылки в рамках; true: цветные ссылки
    linkcolor=red,          % внутренние ссылки
    citecolor=black,        % на библиографию
    filecolor=magenta,      % на файлы
    urlcolor=cyan           % на URL
}

\usepackage{csquotes} % Еще инструменты для ссылок

\usepackage{multicol} % Несколько колонок

\author{Макроэкономика --- 2}
\title{Правила выполнения работ}
\date{}

\begin{document} % конец преамбулы, начало документа

\maketitle



Дорогие студенты второго курса!


На наш взгляд, главной целью домашних работ является не столько контроль знаний, сколько обучение, причем как конкретной экономической дисциплине, так и общим навыкам решения конкретных, практических экономических задач. Какие правила (советы, рекомендации) нам хотелось бы предоставить?

 
Во-первых, решение конкретной задачи предполагает полный ответ на все вопросы, которые поставлены в проблеме. Например, если в задаче просят: <<Напишите бюджетное ограничение $\dots$ нарисуйте график>>, то в вашем решении должно быть как выписано ограничение, так и нарисован график. 

\textbf{\textit{Правило 1: <<Полный балл выставляется за полное решение>>. 
}}  

 
Во-вторых, ваше решение должно содержать ответы на те вопросы, которые вас спрашивают. Это значит, что если, например, просят определить бюджетное ограничение, то надо выписывать формулу для ограничения, а не, скажем, все те формулы, которые, как вам кажутся, могут быть похожи на правильный ответ. При этом из вашего решения должно быть ясно, к чему относится та или иная формула, на какой вопрос задачи она дает ответ.  Если вам кажется, что вы не до конца понимаете, что конкретно хотят от вас в задаче, спрашивайте --- будем стараться отвечать.

 \textbf{\textit{Правило 2: <<Ваш ответ должен быть на вопрос задачи, а не какой-нибудь другой, быть может, даже очень похожий>>.
}}  



В-третьих, ваше решение должно быть обосновано. Необходимо показать, как, используя условия задачи, вы получили ответ на содержащийся в ней вопрос. Если вам потребовалось какое-то дополнительное предположение, обязательно укажите его.  Покажите свою квалификацию как исследователя, т.е. умение проводить логические выводы на основе конкретной информации. Это же замечание касается расчета экономических величин --- используемая вами формула должна содержаться в решении, иначе может создаться впечатление, что вы бездумно переписали ответ у соседа.

 \textbf{\textit{Правило 3: <<Решение должно быть обосновано с приведением всей логической цепочки, соединяющей условия задачи и полученный вами ответ>>.
}} 


В-четвертых, если в условии задачи вводятся какие-то предположения, задумайтесь, почему автор их явно формулирует. Конечно, бывают ситуации, когда в условии задачи содержится лишняя информация, тем не менее, обычно дополнительные ограничения бывают важными для решения проблемы. 

  \textbf{\textit{Правило 4: <<В ходе экономического анализа покажите, как и где вы использовали содержащуюся в задании информацию>>.
}} 

 
 
В-пятых, при ответе на открытый (т.е. требующий развернутого объяснения) вопрос не забывайте приводить сам ответ на этот вопрос. При чем логика построения аналитического текста требует, чтобы ответ на поставленный вопрос содержался либо в начале текста (некоторое заявление, которое вы собираетесь доказывать), либо в конце текста (как вывод из проведенных рассуждений). Только в этом случае у читателя сформируется мнение о вас как о квалифицированном аналитике, способном аргументировано предлагать решение той или иной проблемы. Тоже самое относится к вопросам из категории <<верно/неверно>>, при этом в ответе на такой тип вопросов недостаточно указать, что утверждение неверное, но также требуется кратко написать, почему вы так думаете.

   \textbf{\textit{Правило 5: <<За ходом ваших рассуждений не теряйте ту цель, ради который вы проводите эти рассуждения>>.
}}


 
В-шестых, помните, что при обсуждении тех или иных экономических явлений следует демонстрировать не только знание экономических моделей, но и логику, которую эти модели иллюстрируют. Помните про economic intuition и, по возможности, везде содержательно интерпретируйте полученные на модельном уровне выводы --- это свойство отличает, на наш взгляд, квалифицированного экономиста от человека, знающего экономические модели.   

    \textbf{\textit{Правило 6: <<Модели нужны для объяснения функционирования экономики, а не экономика --- для построения моделей>>. 
}}


 
Конечно, при проверке домашних работ мы можем, как и все люди, допускать ошибки (например, что-то не заметить, или не так сложить сумму баллов). Если что-то остается непонятным, спрашивайте --- мы будем стараться в силу наших возможностей. Помните, что домашние задания вы делайте и сдаете не для нас, а для себя, чтобы научиться. А мы, в свою очередь, чтобы помочь вам это сделать. Система оценивания всегда несовершенна, но она должна выполнять две цели --- выявлять потенциал, а также задавать стимулы. Мы стараемся оценивать так, чтобы у вас были стимулы стать квалифицированными экономистами.

 
Если у вас есть сомнения (возражения, замечания), готовы их выслушать и, по возможности, учесть. Готовы на компромиссы :)

 
Наконец, у нас для вас есть организационные пожелания:

\begin{enumerate}
	\item Скрепляйте работы степлером. 
\item Дедлайн на выполнение одной домашки --- ровно одна неделя, т.е. перед началом следующего семинара вы должны будете сдать работу (если вы вдруг пропускаете семинар, то можете заранее прислать работу своему семинаристу на почту либо через специальную гугл форму). Дедлайн жесткий, т.е. все работы, сданные позже срока, получают $0$ баллов. 
\item  Делайте работы самостоятельно. Даже если мы ничего не пишем, это не значит, что мы ничего не видим.
\end{enumerate}


 
Удачи в ходе дальнейшего обучения в Академии!  

       



\end{document} % конец документа


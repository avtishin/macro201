\documentclass[11pt, a4paper]{article}
%\usepackage{geometry}
\usepackage[inner=1.5cm,outer=1.5cm,top=2.5cm,bottom=2.5cm]{geometry}
\pagestyle{empty}
\usepackage{graphicx}
\usepackage{fancyhdr, lastpage, bbding, pmboxdraw}
\usepackage[usenames,dvipsnames]{color}
\definecolor{darkblue}{rgb}{0,0,.6}
\definecolor{darkred}{rgb}{.7,0,0}
\definecolor{darkgreen}{rgb}{0,.6,0}
\definecolor{red}{rgb}{.98,0,0}
\usepackage[colorlinks,pagebackref,pdfusetitle,urlcolor=darkblue,citecolor=darkblue,linkcolor=darkred,bookmarksnumbered,plainpages=false]{hyperref}
\renewcommand{\thefootnote}{\fnsymbol{footnote}}

\pagestyle{fancyplain}
\fancyhf{}
\lhead{ \fancyplain{}{ Macroeconomics -- II } }
%\chead{ \fancyplain{}{} }
\rhead{ \fancyplain{}{\today} }
%\rfoot{\fancyplain{}{page \thepage\ of \pageref{LastPage}}}
\fancyfoot[RO, LE] {page \thepage\ of \pageref{LastPage} }
\thispagestyle{plain}

%%%%%%%%%%%% LISTING %%%
\usepackage{listings}
\usepackage{caption}
\DeclareCaptionFont{white}{\color{white}}
\DeclareCaptionFormat{listing}{\colorbox{gray}{\parbox{\textwidth}{#1#2#3}}}
\captionsetup[lstlisting]{format=listing,labelfont=white,textfont=white}
\usepackage{verbatim} % used to display code
\usepackage{fancyvrb}
\usepackage{acronym}
\usepackage{amsthm}
\VerbatimFootnotes % Required, otherwise verbatim does not work in footnotes!



\definecolor{OliveGreen}{cmyk}{0.64,0,0.95,0.40}
\definecolor{CadetBlue}{cmyk}{0.62,0.57,0.23,0}
\definecolor{lightlightgray}{gray}{0.93}



\lstset{
%language=bash,                          % Code langugage
basicstyle=\ttfamily,                   % Code font, Examples: \footnotesize, \ttfamily
keywordstyle=\color{OliveGreen},        % Keywords font ('*' = uppercase)
commentstyle=\color{gray},              % Comments font
numbers=left,                           % Line nums position
numberstyle=\tiny,                      % Line-numbers fonts
stepnumber=1,                           % Step between two line-numbers
numbersep=5pt,                          % How far are line-numbers from code
backgroundcolor=\color{lightlightgray}, % Choose background color
frame=none,                             % A frame around the code
tabsize=2,                              % Default tab size
captionpos=t,                           % Caption-position = bottom
breaklines=true,                        % Automatic line breaking?
breakatwhitespace=false,                % Automatic breaks only at whitespace?
showspaces=false,                       % Dont make spaces visible
showtabs=false,                         % Dont make tabls visible
columns=flexible,                       % Column format
morekeywords={__global__, __device__},  % CUDA specific keywords
}

%%% Работа с русским языком
\usepackage{cmap}					% поиск в PDF
\usepackage{mathtext} 				% русские буквы в формулах
\usepackage[T2A]{fontenc}			% кодировка
\usepackage[utf8]{inputenc}			% кодировка исходного текста
\usepackage[english,russian]{babel}	% локализация и переносы
\usepackage{longtable}

%%%%%%%%%%%%%%%%%%%%%%%%%%%%%%%%%%%%
\begin{document}

\begin{center}
{\Large \textsc{Макроэкономика --- 2}}
\end{center}
\begin{center}
Осень 2018
\end{center}
%\date{September 26, 2014}

\begin{center}
\rule{6in}{0.4pt}
\begin{minipage}[t]{.75\textwidth}
\begin{tabular}{llcccll}
\textbf{Лектор:} & Кирилл Сосунов & & &  & \textbf{Email:} & \href{mailto:XYZ@email.org}{XYZ@email.org} \\
\textbf{Семинарист:} &  Диана Петрова & & & & \textbf{Email:} & \href{mailto:diana.petrova.1993@mail.ru}{diana.petrova.1993@mail.ru} \\ 
\textbf{Семинарист:} &  Александр Тишин & & & & \textbf{Email:} & \href{mailto:atishin@nes.ru}{atishin@nes.ru}
\end{tabular}
\end{minipage}
\rule{6in}{0.4pt}
\end{center}
\vspace{.5cm}
\setlength{\unitlength}{1in}
\renewcommand{\arraystretch}{2}

\noindent\textbf{Страницы курса:} \begin{enumerate}
\item \url{macro.cf}
\item \url{https://phenyard.github.io/macro201/}
\end{enumerate}

\vskip.15in
\noindent\textbf{Office Hours:} После занятия либо по предварительной договоренности. 

\vskip.15in
\noindent\textbf{Литература:} %\footnotemark

\begin{enumerate}

\item	Mankiw N.G. Macroeconomics, 7th edition, Worth Publishers. 2010 или
\item	Mankiw N.G. Macroeconomics, 8th edition, Worth Publishers. 2013
\item	Мэнкью Н.Г. Макроэкономика.  Пер. с англ. - М.: Изд-во МГУ, 1994. — 736 с.
\item	Мэнкью Н.Г., Тейлор М. П. Макроэкономика, 2-е издание. Изд-во Питер, 2016, - 560 с.
\item	Абель Э., Бернанке Б. Макроэкономика - Учебное пособие, 2010, Издательство: «Питер»
\item	Бланшар О. Макроэкономика: учебник / О. Бланшар; [пер. с англ.]; науч. ред. пер. Л. Л. Любимов; Гос. ун-т — Высшая школа экономики. — М.: Изд. дом Гос. ун-та — Высшей школы экономики, 2010. — 671, [1] с  — Перевод изд.: Blanchard Olivier. Macroeconomics. Third Edition. Pearson Education Inc.; Prentice Hall, 2003. — 2000 экз. — ISBN 978-5-7598-0556-4 (в пер.).
\item	Шагас Н.Л., Туманова Е.А. Макроэкономика-2, 2006 Издательство: Издательство МГУ ISBN: 5-7218-0869-1 

\end{enumerate}

\vskip.15in
%\noindent\textbf{Цели курса:}  This course is  primarily designed for graduate students ... 

\noindent \textbf{Предварительный план курса:}\\

\begin{longtable} {|p{1.6cm}|p{7cm}|p{1.6cm}|p{7cm}|}

\multicolumn{2}{c}{Лекции}&\multicolumn{2}{c}{Семинары}\\
\hline
Дата	&Название&	Дата&	Название\\
\hline
07.09.2018&	Макроэкономические данные и повторение базовых определений курса "Макроэкономика-1"&	07.09.2018&	Повторение ключевых терминов курса «Макроэкономика-1»\\
14.09.2018&	Модель общего макроэкономического равновесия в долгосрочном периоде&	14.09.2018&	Макроэкономические данные\\
21.09.2018&	Последствия государственной политики в закрытой экономике в долгосрочном периоде&	21.09.2018&	Модель общего макроэкономического равновесия в долгосрочном периоде\\
28.09.2018&	Базовые принципы функционирования системы денежного обращения&	28.09.2018&	Последствия государственной политики в закрытой экономике в долгосрочном периоде\\
05.10.2018&	Причины и последствия инфляции в долгосрочном периоде&	05.10.2018&	Базовые принципы функционирования системы денежного обращения\\
12.10.2018&	Модель малой открытой экономики в долгосрочном периоде&	12.10.2018&	Причины и последствия инфляции в долгосрочном периоде\\
19.10.2018&	Последствия государственной политики в открытой экономике в долгосрочном периоде. Большая открытая экономика&	19.10.2018&	Модель малой открытой экономики в долгосрочном периоде\\
26.10.2018&	Причины существования безработицы в долгосрочном периоде&	26.10.2018&	Последствия государственной политики в открытой экономике в долгосрочном периоде. Большая открытая экономика\\
02.11.2018&	Контрольная работа&	02.11.2018&	Разбор контрольной работы\\
09.11.2018&	Модели структурной безработицы&	09.11.2018&	Причины существования безработицы в долгосрочном периоде\\
16.11.2018&	Модель экономического роста Солоу&	16.11.2018&	Модели структурной безработицы\\
23.11.2018&	Факторы экономического роста в модели Солоу. Эмпрические закономерности экономического роста&	23.11.2018&	Модель экономического роста Солоу\\
30.11.2018&	Проблема конвергенции. Расчет источников экономического роста. Базовая модель эндогенного экономического роста&	30.11.2018&	Факторы экономического роста в модели Солоу. Эмпрические закономерности экономического роста \\
07.12.2018&	Введение в теорию экономических колебаний&	07.12.2018&	Проблема конвергенции. Расчет источников экономического роста. Базовая модель эндогенного экономического роста \\
14.12.2018&	Постановка модели IS-LM (частичное макроэкономическое равновесие в краткосрочном периоде)&	14.12.2018&	Введение в теорию экономических колебаний\\
21.12.2018&	Применение модели IS-LM для анализа последствий экономической политики в краткосрочном периоде&	21.12.2018&	Постановка модели IS-LM (частичное макроэкономическое равновесие в краткосрочном периоде)\\
\hline
\end{longtable}

% \begin{center} 
% \begin{minipage}{5in}
% \begin{flushleft}
% %Chapter 1 \dotfill ~$\approx$ 3 days \\
% {\color{darkgreen}{\Rectangle}} ~A little of probability theory and graph theory	
% \end{flushleft}
% \end{minipage}
% \end{center}

%\vspace*{.15in}
\newpage
\noindent\textbf{Оценки:}  
\begin{itemize}
\item 20 \% Домашние задания (4 штуки, раз в 2-3 недели, дедлайн каждой домашней работы 1 неделя (до следующего семинара), одно из домашних заданий -- эссе. 
\item 10 \% Десятиминутные тесты в классе (5 штук, даются без предупреждения)
\item  30 \% Контрольная работа 
\item 40 \% Финальный экзамен  
\end{itemize}

Шкала относительная, ваша оценка будет зависеть от общего распределения, т.е. от оценок ваших однокурсников. Лучший результат получает наивысшую оценку, остальные считаются от него. Например, если кто-то получил 80/100 по курсу (и это максимальная оценка), то он получает 10 баллов. Все остальные получают оценки по распределению, оценка зависит от процентиля распределения (подробнее https://ru.wikipedia.org/wiki/Квантиль). То есть, если, например, 40\% составляет 32, а 50\% составляет 43, а у вас набрано 36, то вы получаете 5/10. Есть минимальный балл на 3/5 -- это 30/100. Если у вас за курс набрано меньше 30 баллов, то это автоматически влечет неудовлетворительную оценку.

\noindent\textbf{Процентили:}  
В расчете от минимума в 30 баллов.
\begin{itemize}
\item 96\% - 100\% - 10
\item 87\% - 95\% - 9 
\item 81\% - 86\% - 8
\item 61\% - 80\% - 7
\item 41\% - 60\% - 6
\item 21\% - 40\% - 5
\item 1\% - 20\% - 4
\end{itemize}





\vskip.15in
\noindent\textbf{Важные даты:}
\begin{center} \begin{minipage}{3.8in}
\begin{flushleft}
Midterm \#1      \dotfill Ноябрь 02, 2018  \\
Final Exam       \dotfill Январь , 2019  \\
\end{flushleft}
\end{minipage}
\end{center}




\vskip.15in
\noindent\textbf{Посещение:}  Посещение лекций и семинаров необходимо и ожидается от студентов. 

\vskip.15in
\noindent\textbf{Академическая этика:}  Обман, плагиат, списывание и любые другие нарушения академической этики недопустимы.


%%%%%% THE END 
\end{document} 